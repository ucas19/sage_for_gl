\documentclass[12pt,a4paper]{article}

% 数学相关宏包
\usepackage{amsmath, amssymb, amsthm}
\usepackage{amsfonts}
\usepackage{mathtools}
\usepackage{geometry}
\usepackage{graphicx}
\usepackage[bookmarks = false]{hyperref}
\usepackage{xcolor}
\usepackage{enumitem}
\usepackage[scheme=plain]{ctex}
\usepackage{xcolor}
\usepackage{hyperref}
\makeatletter
\renewcommand{\maketitle}{
    \begin{center}
        \vspace*{-1cm}  % 调整这个值来控制顶部间距
        {\LARGE\bfseries \@title \par}
        \vspace{0.2cm}  % 标题与作者间距
        {\large \@author }
%        \vspace{-0cm}  % 作者与日期间距
        {\normalsize \@date \par}
        \vspace{1cm}    % 日期与正文间距
    \end{center}
}
\makeatother

% 页面设置
\geometry{left=1.5cm,right=1.5cm,top=1.5cm,bottom=1.5cm}

% 超链接设置
\hypersetup{
    colorlinks=true,
    linkcolor=blue,
    urlcolor=blue,
    citecolor=red
}

% 定理环境定义
\newtheorem{theorem}{Theorem}[section]
\newtheorem{lemma}[theorem]{Lemma}
\newtheorem{corollary}[theorem]{Corollary}
\newtheorem{proposition}[theorem]{Proposition}
\newtheorem{definition}{Definition}[section]
\newtheorem{example}{Exalple}[section]
\newtheorem{remark}{Remark}[section]

% 数学符号定义
\newcommand{\R}{\mathbb{R}}
\newcommand{\N}{\mathbb{N}}
\newcommand{\Z}{\mathbb{Z}}
\newcommand{\Q}{\mathbb{Q}}
\newcommand{\C}{\mathbb{C}}
\newcommand{\prl}{pr_{\lambda}}
\newcommand{\prt}{pr_{\theta}}
\newcommand{\fracc}[1]{\frac{#1}{2}}
% 文档信息
\title{Exalples  }
\author{Wei}
\date{\today}

\begin{document}
\maketitle
\section{前言}
我写了一个计算的代码, 使用的是python+sagemath. 代码已上传github: \href{ https://github.com/ucas19/sage_for_character }
\section{举例}

\begin{example}
  文章P37页: $\lambda = (\fracc{1},-\fracc{1},\fracc{1}|\fracc{1} )$.
  %文章P39页: $\lambda = (b,a,-c|c )$ where $ c = a > b+1 $
文章结果是(共有20个): 
\begin{equation} \label{eq:1}
\begin{split}
  P_{\lambda} =& M_{\fracc{1}, -\fracc{1},\fracc{1},\fracc{1}}
  +M_{\fracc{1}, \fracc{1},-\fracc{1},\fracc{1}}
  +M_{\fracc{1}, \fracc{1},\fracc{1},-\fracc{1}}
  +M_{\fracc{1}, \fracc{1},\fracc{1},\fracc{1}}
  +2M_{\fracc{3}, -\fracc{1},\fracc{1},\fracc{3}}
  +2M_{\fracc{3}, \fracc{1},-\fracc{1},\fracc{3}} \\
               &+M_{\fracc{3}, \fracc{1},\fracc{1},-\fracc{3}}
  +M_{\fracc{1}, \fracc{3},-\fracc{1},\fracc{3}}
  +M_{\fracc{1}, -\fracc{1},\fracc{3},\fracc{3}}
  +M_{\fracc{5}, \fracc{1},-\fracc{1},\fracc{5}}
  +M_{\fracc{5}, -\fracc{1},\fracc{1},\fracc{5}}
  +M_{\fracc{1}, \fracc{3},\fracc{1},\fracc{3}}  \\
               &+M_{\fracc{1}, \fracc{1},\fracc{3},\fracc{3}}  
  +3M_{\fracc{3}, \fracc{1},\fracc{1},\fracc{3}}  
  +3M_{\fracc{5}, \fracc{1},\fracc{1},\fracc{5}}  
%  P_{\lambda} = \sum M_{b,c,-c|c} +\sum M_{b,c+1,-c|c+1} + \sum M_{b,c,-c+1|c-1} + M_{c,c,b,c+1} +M_{c,c,-b,c-1}
\end{split}
\end{equation}
我们重现计算过程:
\begin{align*}
P_{\fracc{3}, \fracc{1},\fracc{1},\fracc{1}} 
 =&M_{\fracc{3}, \fracc{1},\fracc{1},\fracc{1}}
 +M_{\fracc{3}, \fracc{1},\fracc{3},\fracc{3}}
 +M_{\fracc{3}, \fracc{3},\fracc{1},\fracc{3}}
 +M_{\fracc{5}, \fracc{3},\fracc{1},\fracc{5}}
 +M_{\fracc{5}, \fracc{1},\fracc{3},\fracc{5}}
\end{align*}
\begin{align*}
  \prl(P_{\fracc{3}, \fracc{1},\fracc{1},\fracc{1}} \otimes S^2V )
    = P_{\lambda}
\end{align*}
但是, 事实上不是这样.
\begin{align*}
P_{\fracc{3}, -\fracc{1},\fracc{3},\fracc{1}} 
=&M_{\fracc{3},\fracc{1},\fracc{3}|\fracc{1}}
+M_{\fracc{3},-\fracc{1},\fracc{3}|\fracc{1}}
+M_{\fracc{3},\fracc{1},\fracc{3}|-\fracc{1}}
+M_{\fracc{3},\fracc{3},\fracc{1}|\fracc{1}} \\
&+M_{\fracc{3},\fracc{3},-\fracc{1}|\fracc{1}}
+M_{\fracc{3},\fracc{3},\fracc{1}|-\fracc{1}}
+M_{\fracc{3},\fracc{3},\fracc{3}|\fracc{3}}
+M_{\fracc{5},\fracc{3},\fracc{3}|\fracc{5}}
\end{align*}
然后我们用这个投射盖计算
\begin{equation}\label{eq:2}
  \begin{split}
  \prl(P_{\fracc{3}, -\fracc{1},\fracc{3},\fracc{1}} \otimes S^2V )
  =&2M_{\fracc{1},\fracc{1},\fracc{1}|\fracc{1}}
+2M_{\fracc{1},\fracc{1},\fracc{3}|\fracc{3}}
+3M_{\fracc{3},\fracc{1},\fracc{1}|\fracc{3}}
+M_{\fracc{1},-\fracc{1},\fracc{1}|\fracc{1}}
+M_{\fracc{1},-\fracc{1},\fracc{3}|\fracc{3}}\\
   &+M_{\fracc{3},-\fracc{1},\fracc{1}|\fracc{3}}
+2M_{\fracc{1},\fracc{1},\fracc{1}|-\fracc{1}} 
+M_{\fracc{1},\fracc{1},\fracc{3}|-\fracc{3}}
+2M_{\fracc{3},\fracc{1},\fracc{1}|-\fracc{3}}
+2M_{\fracc{1},\fracc{3},\fracc{1}|\fracc{3}} \\
   &+M_{\fracc{1},\fracc{1},-\fracc{1}|\fracc{1}}
+M_{\fracc{1},\fracc{3},-\fracc{1}|\fracc{3}}
+M_{\fracc{3},\fracc{1},-\fracc{1}|\fracc{3}}
+M_{\fracc{1},\fracc{3},\fracc{1}|-\fracc{3}}
+M_{\fracc{5},\fracc{1},\fracc{1}|\fracc{5}}
\end{split}
\end{equation}
我们对比公式\eqref{eq:1}和\eqref{eq:2}. 我们知道公式\eqref{eq:1}并不是正确的结果. 

正确结果应该是:
\begin{equation}
  \begin{split}
    P_\lambda = 
&M_{\fracc{1},-\fracc{1},\fracc{1}|\fracc{1}}
+M_{\fracc{3},-\fracc{1},\fracc{1}|\fracc{3}}
+M_{\fracc{1},\fracc{1},\fracc{1}|-\fracc{1}}
+M_{\fracc{1},-\fracc{1},\fracc{3}|\fracc{3}}
+M_{\fracc{3},\fracc{1},\fracc{1}|-\fracc{3}} \\
&+M_{\fracc{1},\fracc{1},\fracc{1}|\fracc{1}}
+M_{\fracc{1},\fracc{1},-\fracc{1}|\fracc{1}}
+M_{\fracc{3},\fracc{1},\fracc{1}|\fracc{3}}
+M_{\fracc{3},\fracc{1},-\fracc{1}|\fracc{3}}
+M_{\fracc{1},\fracc{1},\fracc{3}|\fracc{3}}\\
&+M_{\fracc{1},\fracc{3},-\fracc{1}|\fracc{3}}
+M_{\fracc{1},\fracc{3},\fracc{1}|\fracc{3}}
+M_{\fracc{5},\fracc{1},\fracc{1}|\fracc{5}}
+M_{\fracc{3},\fracc{1},\fracc{1}|\fracc{3}}
  \end{split}
\end{equation}

  
\end{example}



\begin{example} \label{example:2}
文章定理4.3.16(61页)中, $\lambda = (-c-1,c,a|-c-1), \mu = (-c,c,a|-c-1)$ 其中$b+1=-c<a$, 令$\lambda' = (-a,c,c|c+1)$.
文章计算得到(顺序稍有不同):
\begin{equation*}
  \begin{split}
  P_{\mu} = &M_{a,-c,c|-c-1}
 +\fcolorbox{red}{white}{$M_{a,-c,-c|-c-1}$}
+M_{a,c,-c|-c-1}
+M_{-c,c,a|-c-1} \\
  &+M_{-c,a,c|-c-1}
+M_{-c,a,-c|-c-1}
+M_{-c,-c,a|-c-1}
  \end{split}
\end{equation*}
事实上, 计算应得到
\begin{equation*}
  \begin{split}
  P_{\mu} =P_{\sigma\lambda'}
  =& p_{\tau_1w_0,\sigma w_0}(1)M_{\tau_1 \lambda'}
  +p_{\tau_2,\sigma w_0 }(1)M_{\tau_1 \lambda'}
  +p_{\tau_3,\sigma w_0 }(1)M_{\tau_1 \lambda'} 
  +p_{\tau_4,\sigma w_0 }(1)M_{\tau_1 \lambda'} \\
  &+p_{\tau_5,\sigma w_0 }(1)M_{\tau_1 \lambda'}
  +p_{\tau_6,\sigma w_0 }(1)M_{\tau_1 \lambda'}
  +p_{\tau_7,\sigma w_0 }(1)M_{\tau_1 \lambda'} \\
 =&M_{\tau_1 \lambda'}
 + \fcolorbox{red}{white}{$ (1+q)M_{\tau_1 \lambda'}$}
  +M_{\tau_1 \lambda'} 
  +M_{\tau_1 \lambda'} 
  +M_{\tau_1 \lambda'}
  +M_{\tau_1 \lambda'}
  +M_{\tau_1 \lambda'}\\
 =&M_{\tau_1 \lambda'}
 +\fcolorbox{red}{white}{$2M_{\tau_1 \lambda'}$}
  +M_{\tau_1 \lambda'} 
  +M_{\tau_1 \lambda'} 
  +M_{\tau_1 \lambda'}
  +M_{\tau_1 \lambda'}
  +M_{\tau_1 \lambda'}\\
 =&M_{a,-c,c|-c-1}
 +\fcolorbox{red}{white}{$2M_{a,-c,-c|-c-1}$}
+M_{a,c,-c|-c-1}
+M_{-c,c,a|-c-1} \\
  &+M_{-c,a,c|-c-1}
+M_{-c,a,-c|-c-1}
+M_{-c,-c,a|-c-1}
  \end{split}
\end{equation*}
其中,
\begin{align*}
  \sigma &= (w3*w1*w2*w3*w1,w_1), \\
  \tau_1 &= (w2*w3*w1*w2*w3*w2*w1,w1),\\
  \tau_2 &= (w3*w2*w3*w1*w2*w3*w2*w1,w1),\\
  \tau_3 &= (w3*w1*w2*w3*w2*w1,w1),\\
  \tau_4 &= (w3*w1*w2*w3*w1,w1),\\
  \tau_5 &= (w2*w3*w1*w2*w3*w1,w1),\\
  \tau_6 &= (w3*w2*w3*w1*w2*w3*w1,w1),\\
  \tau_7 &= (w3*w2*w3*w1*w2*w3,w1)
\end{align*}



\end{example}


\section{遇到不能计算的情况一个举例}

因为权的数量巨大, 并不好写, 因此仅举一例:

\begin{example}

  $\lambda = (-\fracc{1},-\fracc{3},\fracc{3}|\fracc{3}), \mu = (\fracc{1},-\fracc{3},\fracc{3}|\fracc{3}) $, 
  %令$\lambda' =(-\fracc{3},-\fracc{1},-\fracc{1}|-\fracc{5}) $,
  那么, 首先计算$P_\mu$, 文章P42, 定理4.2.10, (2.4)应该也是有问题的, 跟例\ref{example:2}相似, 正确结果应该是: 
  \begin{equation*}
    \begin{split}
      P_{\mu} =& 
+M_{\fracc{5},\fracc{3},-\fracc{1}|\fracc{5}}
+M_{\fracc{5},\fracc{1},-\fracc{3}|\fracc{5}}
+M_{\fracc{3},\fracc{1},-\fracc{1}|\fracc{1}}
+2M_{\fracc{3},\fracc{3},-\fracc{1}|\fracc{3}}
+M_{\fracc{3},\fracc{1},-\fracc{3}|\fracc{3}} \\
               &+2M_{\fracc{5},\fracc{3},\fracc{1}|\fracc{5}}
+2M_{\fracc{5},\fracc{1},\fracc{3}|\fracc{5}}
+2M_{\fracc{3},\fracc{1},\fracc{1}|\fracc{1}}
+3M_{\fracc{3},\fracc{3},\fracc{1}|\fracc{3}}
+3M_{\fracc{3},\fracc{1},\fracc{3}|\fracc{3}} \\
               &+M_{\fracc{5},-\fracc{3},\fracc{1}|\fracc{5}}
+M_{\fracc{5},-\fracc{1},\fracc{3}|\fracc{5}}
+M_{\fracc{3},-\fracc{1},\fracc{1}|\fracc{1}}
+M_{\fracc{3},-\fracc{3},\fracc{1}|\fracc{3}}
+2M_{\fracc{3},-\fracc{1},\fracc{3}|\fracc{3}} \\
               &+M_{\fracc{3},-\fracc{1},\fracc{5}|\fracc{5}}
+M_{\fracc{1},-\fracc{3},\fracc{5}|\fracc{5}}
+M_{\fracc{1},-\fracc{1},\fracc{3}|\fracc{1}}
+M_{\fracc{1},-\fracc{3},\fracc{3}|\fracc{3}}
+M_{\fracc{3},\fracc{5},-\fracc{1}|\fracc{5}} \\
               &+M_{\fracc{1},\fracc{5},-\fracc{3}|\fracc{5}}
+M_{\fracc{1},\fracc{3},-\fracc{1}|\fracc{1}}
+M_{\fracc{1},\fracc{3},-\fracc{3}|\fracc{3}}
+M_{\fracc{3},\fracc{5},\fracc{1}|\fracc{5}}
+M_{\fracc{1},\fracc{5},\fracc{3}|\fracc{5}} \\
               &+M_{\fracc{1},\fracc{3},\fracc{1}|\fracc{1}}
+2M_{\fracc{1},\fracc{3},\fracc{3}|\fracc{3}}
+M_{\fracc{3},\fracc{1},\fracc{5}|\fracc{5}}
+M_{\fracc{1},\fracc{3},\fracc{5}|\fracc{5}}
+M_{\fracc{1},\fracc{1},\fracc{3}|\fracc{1}}
    \end{split}
  \end{equation*}
然后
\begin{equation*}
  \begin{split}
    \prl(P_\mu \otimes S^2V ) =&(96 items) 
+3M_{\fracc{5},\fracc{3},-\fracc{1}|\fracc{5}}
+3M_{\fracc{5},\fracc{3},\fracc{1}|\fracc{5}}
+M_{\fracc{5},\fracc{1},-\fracc{3}|\fracc{5}}
+M_{\fracc{5},-\fracc{1},-\fracc{3}|\fracc{5}}
+3M_{\fracc{3},\fracc{1},-\fracc{1}|\fracc{1}} \\
                              &+2M_{\fracc{3},-\fracc{1},-\fracc{1}|\fracc{1}}
+4M_{\fracc{3},\fracc{1},\fracc{1}|\fracc{1}}
+M_{\fracc{3},\fracc{1},-\fracc{1}|-\fracc{1}}
+5M_{\fracc{3},\fracc{3},-\fracc{1}|\fracc{3}}
+5M_{\fracc{3},\fracc{3},\fracc{1}|\fracc{3}} \\
                              &+M_{\fracc{3},\fracc{1},-\fracc{3}|\fracc{3}}
+M_{\fracc{3},-\fracc{1},-\fracc{3}|\fracc{3}}
+3M_{\fracc{5},\fracc{1},\fracc{3}|\fracc{5}}
+3M_{\fracc{5},-\fracc{1},\fracc{3}|\fracc{5}}
+3M_{\fracc{3},-\fracc{1},\fracc{1}|\fracc{1}} \\
                              &+2M_{\fracc{3},\fracc{1},\fracc{1}|-\fracc{1}}
+5M_{\fracc{3},\fracc{1},\fracc{3}|\fracc{3}}
+5M_{\fracc{3},-\fracc{1},\fracc{3}|\fracc{3}}
+M_{\fracc{5},-\fracc{3},\fracc{1}|\fracc{5}}
+M_{\fracc{5},-\fracc{3},-\fracc{1}|\fracc{5}} \\
                              &+M_{\fracc{3},-\fracc{1},\fracc{1}|-\fracc{1}}
+M_{\fracc{3},-\fracc{3},\fracc{1}|\fracc{3}}
+M_{\fracc{3},-\fracc{3},-\fracc{1}|\fracc{3}}
+2M_{\fracc{3},-\fracc{1},\fracc{5}|\fracc{5}}
+2M_{\fracc{3},\fracc{1},\fracc{5}|\fracc{5}} \\
                              &+M_{\fracc{1},-\fracc{3},\fracc{5}|\fracc{5}}
+M_{-\fracc{1},-\fracc{3},\fracc{5}|\fracc{5}}
+2M_{\fracc{1},-\fracc{1},\fracc{3}|\fracc{1}}
+M_{-\fracc{1},-\fracc{1},\fracc{3}|\fracc{1}}
+2M_{\fracc{1},\fracc{1},\fracc{3}|\fracc{1}} \\
                              &+M_{\fracc{1},-\fracc{1},\fracc{3}|-\fracc{1}}
+M_{\fracc{1},-\fracc{3},\fracc{3}|\fracc{3}}
+M_{-\fracc{1},-\fracc{3},\fracc{3}|\fracc{3}}
+2M_{\fracc{3},\fracc{5},-\fracc{1}|\fracc{5}}
+2M_{\fracc{3},\fracc{5},\fracc{1}|\fracc{5}} \\
                              &+M_{\fracc{1},\fracc{5},-\fracc{3}|\fracc{5}}
+M_{-\fracc{1},\fracc{5},-\fracc{3}|\fracc{5}}
+2M_{\fracc{1},\fracc{3},-\fracc{1}|\fracc{1}}
+M_{-\fracc{1},\fracc{3},-\fracc{1}|\fracc{1}}
+2M_{\fracc{1},\fracc{3},\fracc{1}|\fracc{1}} \\
                              &+M_{\fracc{1},\fracc{3},-\fracc{1}|-\fracc{1}}
+M_{\fracc{1},\fracc{3},-\fracc{3}|\fracc{3}}
+M_{-\fracc{1},\fracc{3},-\fracc{3}|\fracc{3}}
+M_{\fracc{1},\fracc{5},\fracc{3}|\fracc{5}}
+M_{-\fracc{1},\fracc{5},\fracc{3}|\fracc{5}} \\
                              &+M_{-\fracc{1},\fracc{3},\fracc{1}|\fracc{1}}
+M_{\fracc{1},\fracc{3},\fracc{1}|-\fracc{1}}
+2M_{\fracc{1},\fracc{3},\fracc{3}|\fracc{3}}
+2M_{-\fracc{1},\fracc{3},\fracc{3}|\fracc{3}}
+M_{\fracc{1},\fracc{3},\fracc{5}|\fracc{5}} \\
                              &+M_{-\fracc{1},\fracc{3},\fracc{5}|\fracc{5}}
+M_{-\fracc{1},\fracc{1},\fracc{3}|\fracc{1}}
+M_{\fracc{1},\fracc{1},\fracc{3}|-\fracc{1}}
  \end{split}
\end{equation*}
上面这个结果:
\begin{equation*}
  \begin{split}
\prl(P_\mu \otimes S^2V )
&=\prl(P_{\fracc{3},-\fracc{3},\fracc{3}|\fracc{3}} ) \otimes g )  
=\prl(P_{-\fracc{1},-\fracc{3},\fracc{7}|\fracc{3}}) \otimes g ) 
=\prl(P_{\fracc{1},-\fracc{3},\fracc{5}|\fracc{3}}) \otimes g )  \\
&=\prl(P_{\fracc{1},-\fracc{3},\fracc{5}|\fracc{3}}) \otimes S^2V )
=\prl(P_{\fracc{1},-\fracc{3},\fracc{3}|\fracc{3}} ) \otimes V )
=\prl(P_{-\fracc{1},-\fracc{3},\fracc{5}|\fracc{3}}) \otimes V )
  \end{split}
\end{equation*}
\begin{equation*}
  \begin{split}
\prl(P_{-\fracc{1},-\fracc{3},\fracc{5}|\fracc{1}}) \otimes S^2V ) = 
\prl(P_\mu \otimes S^2V ) + P_{\fracc{1},-\fracc{3},\fracc{3}|-\fracc{3}}
  \end{split}
\end{equation*}
\end{example}
判断这96项的时候, 我没有判断出来, 主要是不能确定$M_{{\fracc{3},-\fracc{1},\fracc{3}|\fracc{3}}}$重数是4还是5. 注意到
\begin{equation*}
  \begin{split}
    P_{{\fracc{3},-\fracc{1},\fracc{3}|\fracc{3}}} =& 
+M_{\fracc{5},\fracc{1},\fracc{3}|\fracc{5}}
+M_{\fracc{3},\fracc{1},\fracc{5}|\fracc{5}}
+M_{\fracc{3},\fracc{1},\fracc{3}|\fracc{3}}
+M_{\fracc{5},-\fracc{1},\fracc{3}|\fracc{5}}
+M_{\fracc{3},-\fracc{1},\fracc{5}|\fracc{5}} 
+M_{\fracc{3},-\fracc{1},\fracc{3}|\fracc{3}}
+M_{\fracc{5},\fracc{3},\fracc{1}|\fracc{5}} \\
                                                    &+M_{\fracc{3},\fracc{5},\fracc{1}|\fracc{5}}
+M_{\fracc{3},\fracc{3},\fracc{1}|\fracc{3}}
+M_{\fracc{5},\fracc{3},-\fracc{1}|\fracc{5}}
+M_{\fracc{3},\fracc{5},-\fracc{1}|\fracc{5}}
+M_{\fracc{3},\fracc{3},-\fracc{1}|\fracc{3}}
  \end{split}
\end{equation*}
并不能判断$P_{{\fracc{3},-\fracc{1},\fracc{3}|\fracc{3}}}$在不在$\prl(P_\mu \otimes S^2V )$里面.




%$\lambda = (a,-c,b|b )$
%
%\section{$b<c\leq a$}
%\subsection{$ b+1 < c $}
%$\prl(P_{a,-c,b+1|b} \otimes V ) = \sum(a,-c,b,b) + \sum(a,-c,b+1,b+1)$
%\subsection{$ b+1 = c $ $\lambda = (a,b-1,b,b) $}
%\subsubsection{ $ a>c $}
%$\prl(P_{a,-b-1,b+1|b} \otimes V ) = \sum(a,-b-1,b,b) + \sum(a,-b-1,b+1,b+1)$
%\subsubsection{ $ a = c $}
%
%$\prl(P_{b+2,-b-1,b+1|b} \otimes V ) = \sum M_(b+2,-b-1,b,b) + \sum M_(b+2,-b-1,b+1,b+1) +\sum M_(b+2,-b-1,b+1,b+2)$


\end{document}



